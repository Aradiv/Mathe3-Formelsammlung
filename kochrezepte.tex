%% This is file uses
%% TUDa-CI -- Corporate Design for TU Darmstadt
%% ----------------------------------------------------------------------------
%%
%%  Copyright (C) 2018--2021 by Marei Peischl <marei@peitex.de>
%%
%% ============================================================================
%% This work may be distributed and/or modified under the
%% conditions of the LaTeX Project Public License, either version 1.3c
%% of this license or (at your option) any later version.
%% The latest version of this license is in
%% http://www.latex-project.org/lppl.txt
%% and version 1.3c or later is part of all distributions of LaTeX
%% version 2008/05/04 or later.
%%
%% This work has the LPPL maintenance status `maintained'.
%%
%% The Current Maintainers of this work are
%%   Marei Peischl <tuda-ci@peitex.de>
%%   Markus Lazanowski <latex@ce.tu-darmstadt.de>
%%
%% The development respository can be found at
%% https://github.com/tudace/tuda_latex_templates
%% Please use the issue tracker for feedback!
%%
%% If you need a compiled version of this document, have a look at
%% http://mirror.ctan.org/macros/latex/contrib/tuda-ci/doc
%% or at the documentation directory of this package (if installed)
%% <path to your LaTeX distribution>/doc/latex/tuda-ci
%% ============================================================================
%%
% !TeX program = lualatex
%%

%! Author = stef9998

% Preamble
\documentclass[
%11pt
ngerman,
accentcolor=9c,% Farbe für Hervorhebungen auf Basis der Deklarationen in den
type=intern,
marginpar=false
]{tudapub}

% Packages
\usepackage[english, main=ngerman]{babel}
\usepackage[autostyle]{csquotes}

\usepackage{amsmath}
%\usepackage{amssymb}
%\usepackage{tikz}
%\usetikzlibrary{tikzmark,calc}
%\usepackage{graphicx}
%\usepackage{amsfonts}
%\usepackage{enumitem}
%\usepackage{mathtools}
%\usepackage{mathrsfs}

% Document
\begin{document}
%    \newtheorem{satz}{Satz}[section]
%    \numberwithin{satz}{subsection}
%    \newtheorem{korolar}[satz]{Korolar}
%    \newtheorem{definition}[satz]{Definition}

    \title{Mathe 3Inf/4Etit Kochrezepte}
    \author{Stef9998}
%    \date{} % Ohne Angabe wird automatisch das heutige Datum eingefügt

    \maketitle
    \tableofcontents
    \newpage

    \stepcounter{section}
    \stepcounter{section}
    \stepcounter{section}
    \stepcounter{section}
    \stepcounter{section}
    \section{Verfahren zur Eigenwert- und Eigenvektorberechnung}
        \stepcounter{subsection}
        \subsection{Vektoriteration}
            \textbf{Einfache Vektoriteration nach von Mises:}\\
            Wir setzen die zu Iterationsmatrix $B$ einfach gleich der gegebenen Matrix $A$
            \begin{equation*}
                B = A
            \end{equation*}
            Um die nächste Iteration von $z$ zu bekommen bedarf es nur dieser Formel, welche man die ganze so lange durchiteriert, bis man bei der gewünschen Zahl angekommen ist.
            \begin{equation*}
                z^{(k+1)} = \dfrac{1}{||Bz^{(k)}||}Bz^{(k)}
            \end{equation*}
            Um nun den Rayleigh-Quotienten von einem beliebigen $k$ (meistens wird das vorletzte benutzt) einfach alles in die Formel einsetzen.
            \begin{equation*}
                R(z^{(k)},B)=\dfrac{(z^{(k)})^H Bz^{(k)}}{(z^{(k)})^Hz^{(k)}}
            \end{equation*}
            Bei reellen Werten entspricht $(.)^H$ der Transponierten.\\
            \\
            \textbf{Inverse Vektoriteration von Wielandt:}\\
            Hier ist $B$ mit einem Shift $\mu$ verschoben, und das Ganze invertiert.
            \begin{equation*}
                B = (A-\mu I)^{-1}
            \end{equation*}
            \textbf{Achtung!} Dabei nicht die Inverse bestimmen, sondern über die DGL
            \begin{equation*}
                (A-\mu I)\hat{z}^{(k+1)} = z^{(k)}\\
            \end{equation*}
            $\hat{z}^{k+1}$ bestimmen und dann normieren zu
            \begin{equation*}
                z^{(k+1)} = \dfrac{\hat{z}^{(k+1)}}{||\hat{z}^{(k+1)}||}
            \end{equation*}
            Um nun den Rayleigh-Quotienten von einem beliebigen $k$ (meistens wird das vorletzte benutzt) einfach alles in die Formel einsetzen.
            \begin{equation*}
                R(z^{(k)}, (A - \mu I)^{-1}) = \dfrac{(z^{(k)})^H\hat{z}^{(k+1)}}{(z^{(k)})^Hz^{(k)}}
            \end{equation*}
            Bei reellen Werten entspricht $(.)^H$ der Transponierten.\\
            Einen Eigenwert $\lambda_i$ (oder Schätzung) erhalten wir dann durch Umstellen von
            \begin{equation*}
                \mu_i = \dfrac{1}{\lambda_i - \mu}
            \end{equation*}
            Wobei $\mu_i$ dem Rayleigh-Quotienten entspricht.
    \newpage
    \stepcounter{section}
    \section{Schätzverfahren und Konfidenzintervalle}
        \stepcounter{subsection}
        \subsection{Maximum-Likelihood-Schätzer}
            $f_\theta(x)$ entspricht der Dichtefunktion. Bei diskreten Werten siehe oben.
            \begin{align*}
                L(\theta; x_1, \dots x_n) &= f_\theta(x_1)f_\theta(x_2)\dots f_\theta(x_n)\\
                ln\left( L(\theta; x_1, \dots x_n) \right) &= \sum_{i=1}^{n}ln\left( f_\theta(x_i) \right)
            \end{align*}
            Teilweise kann man $f_\theta$ durch den $ln$ nochmal in Summen aufteilen, da dann alle Summanden ohne $\theta$ bei der Ableitung rausfliegen.
            \begin{equation*}
                \frac{\partial}{\partial \theta} ln\left( L(\theta; x_1, \dots x_n) \right) =
                \frac{\partial}{\partial \theta} \sum_{i=1}^{n}ln\left( f_\theta(x_i) \right) \overset{!}{=} 0
            \end{equation*}
            Nun nach $\theta$ umstellen. Dies ist dann die eindeutige Nullstelle
            \begin{equation*}
                \hat{\theta}(x_1, \dots, x_n) = ...
            \end{equation*}
            Um zu wissen, dass es ein Maximum ist, muss man noch schauen ob die 2. Ableitung $<0$ ist
            \begin{equation*}
                \frac{\partial^2}{\partial \theta^2} ln\left( L(\theta; x_1, \dots x_n) \right) \overset{!}{<} 0
            \end{equation*}
    \newpage
\end{document}